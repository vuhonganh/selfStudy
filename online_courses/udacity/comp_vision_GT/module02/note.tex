\section{Convolution vs Correlation}
\begin{defi}[Cross-Correlation]
	$F$ is input image, $H$ is the kernel mask, $G$ is output image. Then cross correlation is defined as:
	\begin{align}
	G[i,j] = \sum_{u=-k}^{k}\sum_{v=-k}^{k} H[u,v]F[i+u, j+v]
	\end{align}
	It's denoted as:
	\[ G = H \otimes F\]
\end{defi}
\mydefs{Cross-Correlation}
The effect of cross-correlation is that it flips left-right the kernel mask if we give an impulse image as input (image with only one pixel 1, the rest is 0). 

The convolution does not flip the kernel given impulse image as input:
\begin{defi}[Convolution]
	$F$ is input image, $H$ is the kernel mask, $G$ is output image. The convolution is defined as:
	\begin{align}
	G[i,j] = \sum_{u=-k}^{k}\sum_{v=-k}^{k} H[u,v]F[i-u, j-v]
	\end{align}
	denoted as:
	\[G = H \star F \]
\end{defi}

When the filter ($H$) is symmetric cross-correlation and convolution is the same.

\subsection{Properties of Convolution}
\begin{itemize}
	\item Linear
	\item Shift Invariant: operator behaves the same everywhere, i.e. the value of the output depends on the pattern in the image neighbourhood, not the position of the neighbourhood.
	\item Commutative: $f \star g = g \star f$
	\item Associative: $(f \star g) \star h = f \star (g \star h)$
	\item Identity: unit impulse $e = [.., 0,0,1,0,0,...]$: $f \star e = f $
	\item Differentiation: note that differentiation is also a linear operator. And we have:
	$$ \frac{\partial}{\partial x}(f \star g) = \frac{\partial f}{\partial x} \star g $$
\end{itemize}

\subsection{Edge Detection}
Edge is defined as the discontinuity in the image. This can be:
\begin{itemize}
	\item surface normal discontinuity
	\item depth discontinuity
	\item surface color discontinuity
	\item illumination discontinuity
\end{itemize}
In short, an edge is a place of rapid change in the image. Consider the notion that image is a function then edges correspond to extrema of derivative. Hence, to calculate edge, we can compute the gradient of an image by following formula (discrete version):
\begin{defi}[Image Gradient]
	The gradient of an image with intensity $f(x, y)$ is:
\[ \nabla f = [\frac{\partial f}{\partial x}, \frac{\partial f}{\partial y}] = [ \partial_x f, \partial_y f ] \]
where for discrete data, we can approximate using finite differences:
\[ \frac{\partial f(x, y)}{\partial x} \approx \frac{f(x+1, y) - f(x, y)}{1} \approx f(x+1,y) - f(x,y)\]
\[ \frac{\partial f(x, y)}{\partial y} \approx \frac{f(x, y+1) - f(x, y)}{1} \approx f(x,y+1) - f(x,y)\]
\end{defi}
\mydefs{Image Gradient}

Note that \textbf{the gradient points in the direction of most rapid increase in intensity}, where the direction is $\theta$ the angle between two components and the amount of change is given by the gradient magnitude:
\[  \theta = tan^{-1} \Big( \frac{\partial_y f}{\partial_x f}  \Big)  \]
\[  \| \nabla f  \| = \sqrt{(\partial_x f)^2 + (\partial_y f)^2}  \]

\textbf{In a real world, the image normally has a lot of noise. Then we get the extrema of derivative in many places.} Thus we need to firstly smooth the image, then take the derivative of this blurred image and look for peaks. These peaks will represent edges. Denote $f$ the image, $h$ the filter which smooths the image, we remark the properties of differentiation:
\[ 
\frac{\partial}{\partial x} (h \star f) = (\frac{\partial}{\partial x} h) \star f
\]

But because we look for the peak of the result above, so we should consider again a derivative of the above result. This will be:
\[ 
\frac{\partial^2}{\partial x^2}(h \star f) = (\frac{\partial^2}{\partial x^2}h) \star f
\]
So we will search for a value zero of this second derivative where nearby there is strong gradient, i.e. where there is a zero-crossing then it will be the extrema. 

The above formula is 1D second derivative. When we work with image, we have to do the \textbf{Laplacian} operator:
\begin{align}
\nabla^2 h = \frac{\partial^2 f}{\partial x^2} + \frac{\partial^2 f}{\partial y^2}
\end{align}
Again, zero-crossings are the edges.
