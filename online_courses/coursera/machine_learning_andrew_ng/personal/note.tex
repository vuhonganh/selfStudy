This is my personal note covering the whole course. This could be some mathematical notations or some piece of frequently used code. 
\section{Octave tricks}
\subsection{\textbf{mean} and \textbf{std} in Octave} \label{subsecMuStdOctave}
\textbf{mean} and \textbf{std} of a vector in Octave are computer without difficulty. But how about matrix? Since every thing in Octave is a matrix (a vector is just a matrix having a dimension equal one) we should be able to compute \textbf{mean} and \textbf{std} of a matrix. Let's take $X$ - the matrix of input in \eqref{form:matX} as an example.
\begin{align*}
X &= \begin{pmatrix}
x_0^{(1)} & x_1^{(1)} & x_2^{(1)} & ... & x_n^{(1)} \\
x_0^{(2)} & x_1^{(2)} & x_2^{(2)} & ... & x_n^{(2)} \\
...       & ...       & ...       & ... & ...\\
x_0^{(m)} & x_1^{(m)} & x_2^{(m)} & ... & x_n^{(m)}
\end{pmatrix}
\end{align*}

\textbf{mean} and \textbf{std} of X will be computed \textbf{column by column} as below:
\begin{align}
mean(X) &= \begin{pmatrix}
\overline{x_0} & \overline{x_1} & ... & \overline{x_n}
\end{pmatrix} \nonumber \\
std(X) &= \begin{pmatrix}
\sigma_0 & \sigma_1 & ... & \sigma_n
\end{pmatrix}
\end{align}
\myaligns{\textbf{mean} and \textbf{std} of matrix in Octave}

\subsection{Feature Scaling in Octave}
In subsection \ref{subsecMuStdOctave}, we saw how to compute a vector \textbf{mean} and a vector \textbf{std} in Octave. In feature scaling, for each feature, we need to subtract its \textbf{mean} and divide it by its \textbf{std}. It can be written in form of matrix operation if we have a proper matrix \textbf{mean} and a matrix \textbf{std} which has the same size as $X$ with all identical values in each column. They could be done by a matrix multiplication with a $m \times 1$ ones-matrix:
\begin{align}
matMean(X) = \begin{pmatrix}
1 \\
1 \\
... \\
1 \\
\end{pmatrix} \times \begin{pmatrix}
\overline{x_0} & \overline{x_1} & ... & \overline{x_n}
\end{pmatrix}
\end{align}
The listing \ref{lstFeatScale} below shows how we can do it in code:
\begin{lstlisting}[label=lstFeatScale, caption=Feature Scaling in Octave]
mu = mean(X);    % returns a row vector
sigma = std(X);  % returns a row vector

m = size(X, 1);  % returns the number of rows in X

% Etablish matrix of mean and matrix of std having same size as X
mu_matrix = ones(m, 1) * mu;
sigma_matrix = ones(m, 1) * sigma;

% ./ is elements-wise division operation
X_norm = (X - mu_matrix) ./ sigma_matrix;
\end{lstlisting}