This is my personal note covering the whole course. This could be some mathematical notations or some piece of frequently used code. 
\section{\textbf{mean} and \textbf{std} in Octave}
\textbf{mean} and \textbf{std} of a vector in Octave are computer without difficulty. But how about matrix? Since every thing in Octave is a matrix (a vector is just a matrix having a dimension equal one) we should be able to compute \textbf{mean} and \textbf{std} of a matrix. Let's take $X$ - the matrix of input in \eqref{form:matX} as an example.
\begin{align*}
X &= \begin{pmatrix}
x_0^{(1)} & x_1^{(1)} & x_2^{(1)} & ... & x_n^{(1)} \\
x_0^{(2)} & x_1^{(2)} & x_2^{(2)} & ... & x_n^{(2)} \\
...       & ...       & ...       & ... & ...\\
x_0^{(m)} & x_1^{(m)} & x_2^{(m)} & ... & x_n^{(m)}
\end{pmatrix}
\end{align*}

\textbf{mean} and \textbf{std} of X will be computed \textbf{column by column} as below:
\begin{align}
mean(X) &= \begin{pmatrix}
\overline{x_0} & \overline{x_1} & ... & \overline{x_n}
\end{pmatrix} \nonumber \\
std(X) &= \begin{pmatrix}
\sigma_0 & \sigma_1 & ... & \sigma_n
\end{pmatrix}
\end{align}
\myaligns{\textbf{mean} and \textbf{std} of matrix in Octave}