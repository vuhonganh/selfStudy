This is the week 1 of the course Probabilistic Graphical Models (pgm) by Prof. Daphne Koller hosted on Coursera. The week 1 covers quite a lot of notions from distribution to Bayesian Network.

\section{Distribution}
A probability distribution function (aka PDF, probability density function, probability function, or density) is a function that indicates the probability that a given random variable will take on a particular value. If a random variable is discrete (i.e. the value of the random variable is contained in a countable set of values), then the probability density function, $f(x)$ of a random variable $X$ is: $f(x) = P(X = x)$.

The multivariate form of a probability distribution function is the probability that a list of random variables will take on a list of values. If the random variables are discrete, the \textbf{joint probability density function}, $f(x_1, x_2, …, x_n)$ for random variables $X_1, X_2, ..., X_n$ is defined by: 
\begin{align}
f(x_1, x_2, \ldots, x_n) = P(X_1=x_1, X_2=x_2, \ldots, X_n=x_n)
\end{align}
\myaligns{Joint Distribution}

\section{Conditioning}
It's when we condition on some variables. The data obtained after conditioning is called unnormalized measures and we can make them normalized by divide each of them by their sum. For example, we have a joint distribution $P(I, D, G)$ and we condition on variable $G$ to take only its value $G = g_1$. So we need to delete all measures having $G \neq g_1$. This step is called \textbf{reduction} in conditioning. After this step, we obtain measures that does not sum to 1. So we need to sum them up and divide each measure to this sum in order to obtain a (conditional) probability. This step is called \textbf{renormalization} in conditioning.  

\begin{align}
P(I, D, G) \xrightarrow{reduction} P(I, D, G=g_1) \xrightarrow{renormalize} P(I, D, g_1)
\end{align}
\myaligns{Conditioning}

\section{Marginalization}
It's when we have a large data set and we want to observe it only in terms of some variables. For example we have a joint distribution $P(I, D)$ and we are interested only in variable $D$. So we \textbf{marginalize} $I$ by summing over $I$ for each value that $D$ takes: 
\begin{align}
P(D) = \sum_{I} P(I, D)
\end{align}
\myaligns{Marginalization}


\section{Factors}
A factor is a function or a table or a mapping from every assignment of arguments to a real value. We define below factor $\Phi(X_1, \ldots, X_k)$ where $(X_1, \ldots, X_k)$ which is the scope (a set of random variables).
\begin{align}
\Phi: Val(X_1, \ldots, X_k) \rightarrow \mathbb{R}
\end{align}
\myaligns{Factor Definition}

\subsection{Examples of Factor}
Hence, according to the definition above, a joint distribution is a factor. Figure \ref{w1JointDistri} illustrates a joint distribution $P(I,D,G)$ where $I$, $D$, $G$ represents intelligence of a student $(0, 1)$, difficulty of a course $(0,1)$, and the final grade $(A,B,C)$ that student got from that course respectively.
\begin{figure}[!ht]
\centering
\includegraphics[scale = 0.3]{w1JointDistri}
\caption{Joint Distribution}
\label{w1JointDistri}
\end{figure}

Another example is Unnormalized Measure $P(I,D,g^1)$ which has scope $(I,D)$ because $G$ is always fixed to $g^1$. Another \textbf{important example} is Conditional Probability Distribution (CPD). Figure \ref{w1CPD} illustrates the \textbf{CPD} $P(G | I, D)$ which means for every combination of values to the variable $I$ and $D$, we have a probability distribution over $G$.  

\begin{figure}[!ht]
\centering
\includegraphics[scale = 0.4]{w1CPD}
\caption{Conditional Probability Distribution (CPD)}
\label{w1CPD}
\end{figure}

\section{Operations on Factors}
\subsection{Factor Products}
If $\Phi_1(A,B)$ and $\Phi_2(B,C)$ are two factors then we compute their product of $\Phi(A,B,C)$ by multiplying $\Phi_1(A,B)\Phi_2(B,C)$ for all common values of $B$ (see figure \ref{w1FactProd}).

\begin{figure}[!ht]
\centering
\includegraphics[scale = 0.3]{w1FactProd}
\caption{Factor Products}
\label{w1FactProd}
\end{figure}

\subsection{Factor Marginalization}
That's when we want to reduce the scope. For example, we reduce scope $(A,B,C)$ to $(A,C)$ by summing over $B$ for every assignment of $(A,C)$ (figure \ref{w1FactMarginal}).

\begin{figure}[!ht]
\centering
\includegraphics[scale = 0.35]{w1FactMarginal}
\caption{Factor Marginalization}
\label{w1FactMarginal}
\end{figure}

\subsection{Factor Reduction}
That's when we fix a random variable in the scope by one value (in its set of values). For example, $\Phi(A,B,C)$ is reduced to $\Phi(A,B | C = c^1)$ in illustration \ref{w1FactReduce}.
\begin{figure}[!ht]
\centering
\includegraphics[scale = 0.35]{w1FactReduce}
\caption{Factor Reduction}
\label{w1FactReduce}
\end{figure}

\section{Semantics And Factorization}
\subsection{The Student Example}
The student example involving the situation where students take a course. It contains the following random variables:
\begin{itemize}
	\item Course \textbf{D}ifficulty $(D)$ ($0$ = not difficult, $1$ = difficult)
	\item Student \textbf{I}ntelligence $(I)$ ($0$ = not intelligent, $1$ = intelligent)
	\item \textbf{G}rade $(G)$ ($1$ = A, $2$ = B, $3$ = C)
	\item Student \textbf{S}AT score $(S)$ ($0$ = not good, $1$ = good)
	\item Reference \textbf{L}etter from the prof. of this course $(L)$ ($0$ = not referred, $1$ = referred)
\end{itemize}

The dependency graph is shown in figure \ref{w1graphCPD}. Intuitively, we can see the directed edge meaning a strong relation between the 2 variables. \textbf{We also annotate each node of the dependency graph to a CPD (Conditional Probability Distribution)}. \textit{NB: I do not know where this comes from, maybe we can calculate them from the data set.} 

In this example, we have 5 nodes so we will have 5 CPD as shown in figure \ref{w1graphCPD}. We have the chain rule for Bayesian Networks in this example is described in formula below.
\begin{align}
P(G|I,D)P(S|I)P(L|G)P(D)P(I) = P(D,I,G,S,L)
\end{align}
\myaligns{Chain Rule for Bayesian Networks}
Note that this formula is derived from the standard chain rule of probability and some assumption about the independence as following:
\begin{align*}
P(I,D) 		 	&= P(D) P(I|D) = P(I) P(D)\\  
P(D,I,G) 	 	&= P(I,D) P(G|I,D) \\
P(S,D,I,G)   	&= P(D,I,G) P(S|D,I,G) = P(D,I,G) P(S|I) \\
P(L,S,D,I,G) 	&= P(S,D,I,G) P(L|S,D,I,G) = P(S,D,I,G) P(L|G) \\
				&= P(I) P(D) P(G|I,D) P(S|I) P(L|G)
\end{align*}


\begin{figure}[!ht]
\centering
\includegraphics[scale = 0.3]{w1graphCPD}
\caption[Dependency Graph]{Dependency Graph: $D \rightarrow G$, $I \rightarrow G$, $I \rightarrow S$, $G \rightarrow L$}
\label{w1graphCPD}
\end{figure}


\subsection{Bayesian Network Definition}
Bayesian Network is:
\begin{itemize}
	\item a directed acyclic graph (DAG) G whose nodes represent the random variables $X_1,..,X_n$
	\item for each node $X_i$, there is a CPD: $P(X_i | Parents_G(X_i))$
	\item represents a joint distribution via the chain rule for Bayesian Networks in formula \ref{formChainRule} 
	\begin{align}\label{formChainRule}
	P(X_1,..,X_n) = \prod_i P(X_i|Parents_G(X_i))
	\end{align}
	\myaligns{Bayesian Network Definition}
\end{itemize}
We can prove that Bayesian Network is a legal distribution meaning it satisfies $P \geq 0$ and $\sum_i P(X_i) = 1$. The first one is trivial. The second one is proved as follow:
\begin{align*}
\sum_{D,I,G,S,L} P(D,I,G,S,L) 	&= \sum_{D,I,G,S,L} P(D)P(I)P(G|I,D)P(S|I)P(L|G) \\
								&= \sum_{D,I,G,S} P(D)P(I)P(G|I,D)P(S|I) \sum_L P(L|G) \\
								&= \sum_{D,I,G,S} P(D)P(I)P(G|I,D)P(S|I) * 1\\
								&= \sum_{D,I,G} P(D)P(I)P(G|I,D) \sum_S P(S|I)\\
								&= \sum_{D,I} P(D)P(I) \sum_G P(G|I,D)\\
								&= ...\\
								&= 1
\end{align*}

Another notation: Let G be a graph over $X_1, .., X_n$. A distribution $P$ is called to \textbf{factorize over graph G} if:
\begin{align*}
P(X_1, .., X_n) = \prod_i P(X_i | Par_G(X_i))
\end{align*}

\section{Reasoning Patterns}

\subsection{Flow of Probabilistic Influence}
We say $X$ influence $Y$ if condition on X changes beliefs about $Y$. Hence, $X$ can influence $Y$ when:
\begin{itemize}
	\item $X \rightarrow Y$
	\item $Y \rightarrow X$
	\item 
\end{itemize}

\subsection{Causal Reasoning}
In a Bayesian network, if there is a path from one random variable to another, then the variable at the root of the path is said to affect the other random variables in the path via causal reasoning. For example, if $A \rightarrow B \rightarrow C$, then $A$ affects $B$ and $C$ via causal reasoning and $P(C)$ is generally \textbf{not equal} to $P(C|A)$.
Intuitively, inference goes in causal direction (direction of edges): \textbf{top down}.

\subsection{Evidential Reasoning}
In a Bayesian network, if there is a path from one random variable to another, then the variable at the end of the path is said to affect the other random variables in the path via evidential reasoning. For example, if $A \rightarrow B \rightarrow C$, then $C$ affects $A$ and $B$ via evidential reasoning and $P(A)$ is generally not equal to $P(A|C)$.
Bottom up: Condition the result, ask what the probability of the initial variables was (back from the cause), using Bayes' rule.

\subsection{Inter-causal Reasoning}
Flow of information between (for example) two causes of a single effect. When you condition the result, the causes are \textbf{no longer independent}. This also works across several edges and nodes. Don't really understand "explain away"! 
